\section{Introduction}
In this report, we explore Van Emde-Boas (vEB) trees and their efficiency in reducing cache memory misses during search operations compared to plain binary search trees (BSTs). The vEB tree is a sophisticated data structure that allows for extremely fast operations on a set of integers. This efficiency is achieved through a hierarchical structure that reduces the complexity of operations to $O(\log \log M)$, where $M$ is the size of the universe. In contrast, plain BSTs, which are simpler in design, offer $O(\log N)$ complexity for balanced trees, where $N$ is the number of elements.

We aim to implement both data structures, measure their performance in terms of cache misses using a specialized tool, and compare the results across various configurations. Our hypothesis is that the hierarchical nature of vEB trees will result in fewer cache misses compared to plain BSTs.\\

You can find all the code and documentation for this project on our GitHub repository: \textbf{\url{https://github.com/pierre-jezegou/van-emde-boas-tree}}.